\documentclass[11pt]{article}

\newcommand{\numpy}{{\tt numpy}}    % tt font for numpy

\topmargin -.5in
\textheight 9in
\oddsidemargin -.25in
\evensidemargin -.25in
\textwidth 7in

\begin{document}

% ========== Edit your name here
\author{Daniel López - Heber Orellana - Anderson Peña} \date{\today}
\title{Map 1: Curvas de Bezier}
\maketitle

\medskip

% ========== Begin answering questions here
\begin{enumerate}

\item
Curva de bezier de los puntos de control: $P_0(4,1)$, $P_1(28,48)$, $P_3(50,42)$, $P_4(40,5)$
% ========== Just examples, please delete before submitting 
Grafica:
\begin{figure}
\includegraphics[scale=25]{grafica1.png}
\caption{Curva de bezier y puntos}
\label{fig: grafica1}
\end{figure}
\begin{equation}
    a^2 + b^2 = c^2.
\end{equation}


\begin{equation}
    \mathbf{A} \mathbf{x} = \mathbf{b}.
\end{equation}

An example of a matrix \LaTeX:
\begin{equation}
    \mathbf{A} = \left(
    \begin{array}{ccc}
    3 & -1 & 2 \\ 	
    0 & 1 & 2 \\ 
    1 & 0 & -1 \\
\end{array} 
\right).  
\end{equation}

With a labeled equation such as the following:
\begin{equation}
    \label{accel}
    \frac{d^2 x}{d t^2} = a
\end{equation}
you can referrer to the equation later. In equation \ref{accel} we defined acceleration.
% ========== END examples


\item
Answer to question 2

% ========== Continue adding items as needed

\end{enumerate}

\end{document}
\grid
\grid